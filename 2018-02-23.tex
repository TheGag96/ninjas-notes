\section{2018-02-23}

\subsection{A ``review'' of vector analysis}

\subsubsection{Vector spaces}

\underline{Definition:} a \textbf{vector space} is a set of \textbf{vectors} and \textbf{scalars} along with the two operations $\oplus$ and $\odot$ that adhere to the following:

\begin{enumerate}
  \item A vector space, $V$, is closed under vector addition $\oplus$ and $\odot$

  \begin{enumerate}
    \item $\alpha \oplus \beta = \beta \oplus \alpha;\ \forall \alpha,\ \beta \in V$
    \item $(\alpha \oplus \beta) \oplus \gamma = \alpha \oplus (\beta \oplus \gamma);\ \forall \alpha,\ \beta,\ \gamma \in V$
    \item $\forall \varnothing \in V \text{(``zero'' vector)} \ni \alpha \oplus \varnothing = \alpha;\ \forall \alpha \in V$
    \item $\forall \alpha \in V,\ \exists \beta \in V \ni \alpha \oplus \beta = \varnothing;\ \beta = -\alpha$
  \end{enumerate}

  \item V is closed under scalar multplication (if $\alpha \in V$ and c is a scalar)

  \begin{enumerate}
    \item $c \odot (\alpha \oplus \beta) = c \odot \alpha \oplus c \odot \beta;\ \forall \alpha,\ \beta \in V$
    \item $(c \oplus d) \odot \alpha = c\odot\alpha \oplus d\odot \alpha$
    \item $(c \odot d) \odot \alpha = c \odot (d \odot \alpha)$
    \item $1 \odot \alpha = \alpha$
  \end{enumerate}
\end{enumerate} 

Examples (are they vector spaces?):

\begin{enumerate}
  \item Let $V$ be the set of all pairs of real numbers ($\mathbb{R} ^ 2$) and the scalars the real numbers ($\mathbb{R}$) and let $\oplus$ be defined as the sum of two vectors is the vector whose components are the ``usual'' sum of the operands and scalar multiplication is the ``usual'' multiplication. \textrightarrow\ \underline{yes}

  \item $V$ is the set of all functions on $[0,1]$ \textrightarrow\ \underline{yes}

  \item Let asdf the set of matrices $A$ of real matrices \textrightarrow\ \underline{yes} <Note: Could not understand what Price wrote>

  \item Let $V = \mathbb{Z}^\oplus$, scalars a reals. \textrightarrow\ \underline{no} <Note: Could not understand what Price wrote>
\end{enumerate}


\subsubsection{Subspaces}

A subset of the vectors $v \in V$ is a \textbf{subspace} if that subset is closed under $\oplus$ and $\odot$.

\underline{Definition:} a \textbf{linear combination} is any sum of constant (scalar) multiples of vectors in the vector space.

\underline{Definition:} Let $S = {\alpha_1, ..., \alpha_n} \subseteq V$ if any $v \in V$ is some linear combination of the vectors in $S$. Then $S$ is said to \textbf{span} $V$.

\underline{Definition:} A set $S = {\alpha_1, ..., \alpha_n} \subseteq V$ is said to be \textbf{linearly dependent} if $\exists a_i, 1 \leq i \leq n$, not all 0, such that $a_1 \alpha_1 + a_2 \alpha_2 + .. + a_n \alpha_n = 0$.

If for $a_1 \alpha_1 + a_2 \alpha_2 + .. + a_n \alpha_n = 0$ the solution is $a_1 = a_2 = .. = a_2 = 0$, then $S$ is said to be \textbf{linearly independent}.

\underline{Definition:} The \textbf{dimension} of $V$ ($dim\ V$) is the number of vectors in a  linearly independent spanning set of $V$.

\underline{Definition:} If a set $S = {\alpha_1, ..., \alpha_n}$ spans the vector space $V$ and is linearly independent, it is caleld a \textbf{basis}.
\section{2018-01-19}


\subsection{Make and makefiles}

\shellcmd{make} is a Unix command. It's part of a utility that's designed to lead to easier building of programs the goal is to simplify the process. Many times, the creation of an executable is a very quick, simple, easy (cheesy) compile command. With just a few files, that suffices. But, many times, projects are comprised  of more than just a few files. Also, you may change just one file, there's no need to rebuild all the files in the project -- only the file(s) that have been changed or are dependencies of those files. \shellcmd{make} has the ability to accomplish this. It uses the time stamps and it knows the dependencies.

\subsubsection{The basics}

\begin{itemize}
  \item "makefile" or "Makefile" - a file you create that is executed by the command \shellcmd{make}
  \item \shellcmd{make} - the command
  \item target - the executable that is to be built
  \item dependencies - the files used to build the target
\end{itemize}

Motivation is to expedite the build to save time.

\subsubsection{Basic example makefile}

\begin{lstlisting}[language=make]
# This is the makefile for the stooge project.
stooge : moe.o larry.o curly.o shemp.h 
  g++ moe.o larry.o curly.o -o stooge

moe.o : moe.cpp larry.cpp shemp.h curly.cpp 
  g++ -c moe.cpp curly.cpp larry.cpp -o moe.o 

curly.o : curly.cpp shemp.h
  g++ -c curly.cpp -o curly.o

larry.o : larry.cpp shemp.h
  g++ -c larry.cpp -o larry.o
\end{lstlisting}

\textit{Note, requires tabs for indentation}

\subsubsection{A more robust one}

\begin{lstlisting}[language=make]
# makefile for stooge
cxx = /mst/bin/g++
FLAGS = -W -Wall -s -pediantic -errors
OBJECTS = moe.o \
          larry.o \
          curly.o
.SUFFIXES : .cpp
.cpp .o : $<
  ${cxx} -c ${FLAGS} $< -o $@
default : all
all : stooge
stooge : ${OBJECTS}
  ${cxx} ${FLAGS} ${OBJECTS} -o $@
clean : 
  -@rm -f core > /dev/null 2>&1
  -@rm -f stooge > /dev/null 2>&1
\end{lstlisting}

\subsubsection{Basic rules}

\begin{enumerate}
  \item There must be a tab at the lead or beginning of a command line. You can use the Unix command:

  \begin{lstlisting}[language=make]
  cat -v -t -e makefile
  \end{lstlisting}

  Will print a \shellcmd{\^I} anywhere there is a tab char, and a \shellcmd{\$} at the end of every line.

  "When using Makefiles, you must use tabs to indent. In jpico, when you press tab, it puts something that \textit{looks} like a tab. But it's not a tab. It took me about 3 or 4 hours to figure it out. Jpico is named after a guy named Joe. I looked him up, and he died several years earlier, but I dug him up and killed him again." --Captain Price, 2018

  \item \shellcmd{\textbackslash} at the end of a line continues that line on the next line of the file (no space between the \shellcmd{\textbackslash} and the newline char)
  \item Don't use tabs anywhere else.
  \item \shellcmd{make} ignores blank lines. \shellcmd{\#} begins a comment. 
  \item Targets without dependencies are allowed.
  \item Command lines are echoed to the standard out
  \item \shellcmd{-@} will suppress that echo.
\end{enumerate}

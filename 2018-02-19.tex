\section{2018-02-16}

\subsection{The copy/swap maneuver}

...for the overload of the assignment operator. In our array class, we had an overload of the assignment operator - that was this:

\begin{lstlisting}[language=C++] 
Array& Array::operator=(const Array& s) {
  if (ptrToData != s.ptrToData) {
    setSize(s.m_size);
    for (int i = 0; i < m_size; i++) {
      ptrToData[i] = s.ptrToData[i];
    }
  }

  return *this;
}
\end{lstlisting}

Can we do better? Problems with that code:

\begin{enumerate}
  \item To make allowance for self-assignment
  \item Time spent wakling down the array
\end{enumerate}

Answer:

\begin{lstlisting}[language=C++] 
Array& Array::operator=(Array s) {
  swap(*this, s);
  return *this;
}
\end{lstlisting}

Way Kool\textsuperscript{\$\textregistered\texttrademark} code.

Most important! is that you undertsand the swap used here. You need to write your own swpa function.

\begin{lstlisting}[language=C++] 
class Array {
  public:
    //...
    friend void swap(Array& a1, Array& a2) {
      std::swap(a1.size, a2.size);
      std::swap(a1.ptrToData, a2.ptrToData);
    }
}
\end{lstlisting}

We can't use the std::swap to swap the arrays because it uses the copy constructor and the assignment operator - that which we are trying to overload! SO, in our new version of \cpp{operator=}, we use \underline{pass-by-value} letting the compiler make the copy for us usingour copy constructor. If it fails, then \cpp{*this} is not affected. And, when the copy is made, we just steal the ptr, as we did with a move constructor. The passed object goes out of scope and we have the goods. 

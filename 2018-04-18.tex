%
%  notes.tex
%  Desktop
%
%  Created by Illya Starikov on 04/18/18.
%  Copyright 2018. Illya Starikov. All rights reserved.
%

\section{2018-04-18}

\subsection{Curiously Recurring Template Pattern}

\subsubsection{Definition}

This is also known as the Barton-Nackman Trick.

Some applications of the Curiously Recurring Template Pattern:

\begin{itemize}
  \item Static Polymorphism
  \begin{itemize}
    \item No Virtual Table
    \item Objects use less RAM
    \item Function Call Much Faster
  \end{itemize}
\end{itemize}

However, some cons include:

\begin{itemize}
  \item More complex
  \item Results in bigger executable
  \item Have to know the types at compile time
  \begin{itemize}
    \item For example, \cpp{Vector<AbsMatrix&>} won't work
  \end{itemize}
\end{itemize}

Also, mixins!

\begin{itemize}
  \item Provide common implementation of features
\end{itemize}

\subsubsection{Implementation}
Our interface looks like:

\begin{lstlisting}[language=C++]
template <class Derived>
class Base {
  public:
    int foo() {
      return asDerived.foo();
    }

    Derived& asDerived {
      return static_cast<Derived&>(*this);
    }
};

class Seven : public Base<Seven> {
  public:
    int foo() {
      return 6;
    }
};

Seven five;
five.foo(); // works fine

Base<Seven> five;
five.foo(); // also works fine
\end{lstlisting}

So, say we want to use this elsewhere

\begin{lstlisting}[language=C++]
template <class Derived>
void printFoo(Base<Derived> b) {
  std::cout << b.foo() << "\n";
}

Base<Seven> s;
Base<Eleven> e;

printFoo(s);
printFoo(e);
\end{lstlisting}

